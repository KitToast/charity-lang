\documentclass[12pt]{article}

\begin{document}

\section{MM Proof tool}
There are three versions of the modal proof tool.
\begin{enumerate}
\item One finds proofs in Hennessy Milner Logic, (no recursion)
\item one finds proofs in Modal Mu (with recursion)
\item one finds proofs in Modal Mu with ``types'' (independence of actions)
\end{enumerate}

To run the proof tool use the following command:

``proveIt mmExample''

The files ``testSh, hmlSml.sparc-solaris'' must be in the current directory.

Create your own script from the hmlExample file.




\section{Commands}
The commonly used commands in the MM tool are:

\begin{itemize}
\item[P] parses a back quoted string and returns an HML proposition
\item[$\vdash$] uses the characters ``$\mid$'' and ``-'' to
      turn a pair of lists of propositions into an assumption
\item[xdviProofOrCe] takes an assumption (or judgement) and returns
     either a proof (displayed on xdvi) or an agent (a counter example
     in the form of a tree) which satisfies the antecedent
     and does not satisfy the succedent.
\end{itemize}

Examples:

\begin{verbatim}
xdviProofOrCe ([P `<a><b><c>T`] |- [P `<a><b>T`]);
\end{verbatim}

Displays a proof that, being able to perform an ``a'' action followed by
a ``b'' action, followed by a ``c'' action implies that it is possible to
perform an ``a'' action followed by a ``b'' action.

\begin{verbatim}
xdviProofOrCe ([P `<a><b>T`] |- [P `<a><b><c>T`]);
\end{verbatim}

Displays an agent which can perform an ``a'' action followed by a ``b'' action
but can't subsequently perform a ``c'' action.

\begin{verbatim}
xdviProofOrCe ([P `<a>T \/ <b><c>T`] |- [P `<a>T`, P `<b>T`]);
\end{verbatim}

Displays another proof.  What this really shows is how to use
lists; the succedent is a list of propositions.

These commands are the only ones which I expect that a casual user
might use.  A more complete set follows.

\section{More commands, for the cognescenti}
A Proposition is a HML proposition (And's, Or's, Boxes, Posses, Variables)
(e.g. \begin{verbatim} \Poss{a}\Bx{b}F\AND\Bx{c}F \end{verbatim})
A Sequent is a pair of lists of Propositions.

\begin{verbatim}
(\Poss{a}\Bx{b}F,\Bx{c}F$\vdash$ \Poss{a}\Bx{b}F,\Bx{c}F)
\end{verbatim}

A Rule is one of the lAnd,rOr, ... proof rules.

A Tableau is a proof tree (possibly closed) built
up from a sequent and rules.

\subsection{dp}
The function ``dp'' will perform the decision procedure
on the (cut free) tableau.  It will return a tableau.
It needs a history list; your best bet is to give it an empty list.


\subsection{Rule functions}
Some commands:
\begin{itemize}
\item antecedent : JUDGEMENT $\mapsto$ hml list
\item succedent : JUDGEMENT $\mapsto$ hml list
\item proofFold : (JUDGEMENT $\mapsto$ JUDGEMENT) $\mapsto$ JUDGEMENT $\mapsto$ JUDGEMENT
\item isAProof : JUDGEMENT $\mapsto$ bool
\end{itemize}


The following functions apply the rules to a Judgement.
They only make sense at Assumptions, and at Judgements
they are the identity function.  To be useful
the proofFold function should be applied to them.

\begin{itemize}
\item lAnd
\item rOr
\item lFalse
\item rTrue
\item lOr
\item rAnd
\item boxRule
\item possRule
\item idRule
\end{itemize}


Example:

\begin{verbatim}
lAnd ([P /\ Q, <a>T, [b]F] |- DELTA)
\end{verbatim}
will return
\begin{verbatim}
lAnd ([P, Q, <a>T, [b]F] |- DELTA)
\end{verbatim}

A more sensible example would be:
$$ proofFold lAnd tableau; $$

or $proofFold (lAnd o rOr) tableau;$
where ``o'' is the composition operator.

This returns a TABLEAU, and does not print out anything.

The preceeding list of commands either do too little or too much.
Without the proofFold the commands do too little, and with
the proof fold the commands work over every assumption.
To regain a little more control, use the ``at'' function.


$$at : (Tableau \mapsto Tableau) \mapsto int list \mapsto Tableau$$

The function $$at : (proofFold lOr) (0::0::2::nil) tableau$$

will take the left most child (the zero), its leftmost child (the second zero)
and its ``third'' child (the 2) and apply the function (proofFold lOr)
to that.  Thus, only a part of the tree have the subsequent explosion
due to distribution.

\subsection{dualising functions}

\begin{itemize}
\item dualJudgement (TABLEAU $\mapsto$ TABLEAU)
\item dualProp : (PROP $\mapsto$ PROP)
\item dualSeq
\item dualRule	(RULE $\mapsto$ RULE)
\end{itemize}

The function ``dualJudgement'' will take a tableau and return the
dual of it (i.e. flip antecedent and seccedent, flip props to dual props)


\subsection{Cuts}
exception UN\_CUTABLE

\begin{description}
\item[upL]  TABLEAU $\mapsto$  TABLEAU
	tries to push the cut up the left branch.  If the rule isn't a
	cut, then it does nothing.

\item[upR]  TABLEAU $\mapsto$  TABLEAU
	tries to push the cut up the right branch.  If the rule isn't a
	cut, then it does nothing.

\item[upBoth]  TABLEAU $\mapsto$  TABLEAU
	tries to push the cut up both branches (works only for a few rules)
	If the rule isn't a cut (or of the right form) then does nothing.

\item[cutElim]  TABLEAU $\mapsto$  TABLEAU
	Tries upL; if it is unsuccessful tries upR; if it is unsuccessful
	tries upBoth.
\item[cutElimR]  TABLEAU $\mapsto$  TABLEAU
	Tries cutElim with ``fold''; thus, cutElimR at the root of a proof
	tree could affect internal nodes of the tree.
\item[mkCut] PROP *  TABLEAU * TABLEAU $\mapsto$ TABLEAU
	Allows the user to take two proofs and ``cut'' them together.
	e.g. (mkCut (P Q, dp proof1, dp proof2)
	where proof1 and proof2 have Q in the succedent and antecedent
	respectively.  Otherwise raises the exception UN\_CUTTABLE
\end{description}


\subsection{Pretty Print Proofs}

\begin{description}
\item[ppPropText] prop $\mapsto$  string
		takes a prop and pretty prints in in TEXT (not latex)

\item[ppPropLatex] prop $\mapsto$  string
		takes a prop and returns a string, which when
		run through latex will pretty print a proposition
\item[ppJudgement] TABLEAU $\mapsto$ string
		takes a tableau, and return a string, which when
		run through latex will pretty print a proof.

\item[latexProlog] string	used as a prolog for latex and the proof

\item[latexEpilog] string	used as a epilog for latex and the proof

val latexCaption] string $\mapsto$ string
\end{description}

Example:

The following string is a fine latex program:

print (ppProp.latexProlog \^ (ppProp.ppJudgement proof) \^ ppProp.latexEpilog);

So is the following, but it has a caption on the picture.

print (ppProp.latexProlog \^ (ppProp.ppJudgement proof) \^
		(ppProp.caption "A Fine caption") \^ ppProp.latexEpilog);



\subsection{States}

\begin{description}
\item[oops]  exception HAS\_A\_PROOF
\item[counter] TABLEAU $\mapsto$ state
	Finds a counter example for the failed proof;
	returns an agent (state) which satisfies the antecedent and doesn't
	satisfy the succedent.  If there is a proof, it raises an exception.
\item[ppStateText] state $\mapsto$ string
\item[ppStateLatex] state $\mapsto$ string
		Given a counter example (an agent, a state) returns
		some latex which draws the darned thing.
\item[latexProlog] string
\item[latexEpilog] string
\end{description}


Example:
print (State.latexProlog \^ (State.ppStateLatex (State.counter failedProof))
	\^ (State.latexEpilog));

will print out a latex program which will draw the agent.

\end{document}

